\documentclass[a4paper, 12pt]{article}
% Template LaTeX file, which try01.html and try01_latex.aux are ``loosely'' based on.
% Latex-compiling this file yields something similar, but not identical, such as
% the section numbers. Or, try01.html is hand-annotated. Most importantly,
% 'make' generates try01_tmpl.html from this LaTeX file where HTML is corrected
% by this tool, whereas try01.html is before any correction.

\usepackage[main=japanese,english]{babel}
\usepackage[utf8]{inputenc}
\usepackage{amssymb}  % for \mathbb{Z}
\usepackage{amstext}  % for $\text{例}$ or \textrm{例}  % both are included in the ``amsmath'' package.
\usepackage[unicode]{hyperref}
\usepackage{graphicx}

\begin{document}
\tableofcontents  % The initial declaration MUST contain a4paper.

\title{My Title}
\maketitle

\section{はじめに}

\setcounter{section}{4}
\setcounter{subsection}{5}
\subsection{対称的な数}
\label{sec_symmetry_mul}

以下が使える。

\addtocounter{subsection}{1}
\subsection{2乗と3乗}

つまり、計算は単純化される。

\begin{equation}
  (x \pm \epsilon)^2 = x^2 \pm 2x\epsilon + \epsilon^2 \label{eq_square_pm}
\end{equation}

同様に、
\begin{equation}
  (x \pm \epsilon)^3 = x^3 \pm 3x^2\epsilon + 3x\epsilon^2 \pm \epsilon^3 \label{eq_cubic_pm}
\end{equation}

以下が公式(\ref{sec_symmetry_mul}章)の応用例。

\begin{eqnarray}
  84\times 86 & = & (85+1)(85-1) \nonumber\\
              & = & 7224
\end{eqnarray}

ここで、$85^2$は項(式(\ref{eq_square_pm})の$2x\epsilon$)。 

\addtocounter{subsection}{1}
\subsubsection{割り算}

これは、

\begin{eqnarray}
  \frac{1+x}{1-x} & = & \frac{2-(1-x)}{1-x} \nonumber\\
                & = & \frac{2}{1-x} - 1\nonumber\\
                & = & 1 + 2x + 2x^2 + 2x^3 + \cdots \label{eq_approx_symmetric_frac_strict}\\
                & \sim & \frac{1}{1-2x} \label{eq_approx_symmetric_frac2}\\
                & = & 1 + 2x + 4x^2 + 8x^3 + \cdots  \label{eq_approx_symmetric_frac2_series}\\
                & \sim & % 1 + 2x \label{eq_approx_symmetric_frac_1order}
  \left\{
  \begin{array}{ll}
    1 & (x=0)\\
    1 + 2x
  \end{array} \right. \label{eq_approx_symmetric_frac_1order}
\end{eqnarray}

近似(\ref{eq_approx_symmetric_frac2})は、、(参考: 式(\ref{eq_approx_symmetric_frac_1order}))だ。したがって、 (式(\ref{eq_approx_symmetric_frac2}))の場合は、近似1 (式\ref{eq_approx_symmetric_frac_1order})の計算。

一例として、
\begin{equation}
 26\div24 = \frac{25+1}{25-1}
\end{equation}
を展開(式(\ref{eq_approx_symmetric_frac_strict}))と、式(\ref{eq_approx_symmetric_frac2_series})で計算。

\begin{table}[htbp]
  \centering
  \caption{25の計算}
  \label{tbl_25pm1_taylor}
  \begin{tabular}{cl}
    \hline
    Equation & Num \\
    \hline
    (\ref{eq_approx_symmetric_frac_strict}) & 1.08 \\
    (\ref{eq_approx_symmetric_frac2_series}) & 2.08 \\
    \hline
  \end{tabular}
\end{table}


\end{document}

